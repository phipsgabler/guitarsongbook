%% As arguments to the class we can specify the output format, and whether the indices should be
%% typeset.
\documentclass[screena4, authorindex]{../guitarsongbook}
\usepackage[utf8]{inputenc}

\begin{document}
%% Default values can be changed by setting the respective pgf keys:
\pgfkeyssetvalue{/gsb/authorindextitle}{Autoren}

%% Title and author indices are created automatically; this can also be controlled by the
%% `authorindex' class option. You need to run the songindex program to get something out of this!
\maketitleindex
\makeauthorindex

\begin{songs}

\begin{song}{House of the rising Sun}[by={Traditional}]
 
%% Here, we can set chords which are used in the piece. The heading text be customized via the pgf
%% key "/gsb/tabsheading".
\begin{tabs}
\gtab{Am}{X03320} \gtab{C}{X32010} \gtab{G}{320003}
\gtab{F}{144311} \gtab{E7}{030200}
\end{tabs}

%% In this environment, we can write instrumental parts; chords are set on the baseline, so this
%% isn't thought for setting lyrics.
\begin{instrumental}{Intro:}
\[Am] \[C] \[D] \[F]
\[Am] \[C] \[E7]
\end{instrumental}

\begin{verse}
There i\[Am]s a h\[C]ouse in \[D]New Or\[F]leans
They \[Am]call the R\[C]ising \[E7]Sun, \[E7]
And it's \[Am]been the ru\[C]in of m\[D]any a poor \[F]boy.
And \[Am]God, I \[E]know I'm \[Am]one. \[E7]
\end{verse}

\begin{verse}
My ^mother ^was a ^tailor. ^
She s^ewed my ^new blue ^jeans. ^
My ^father ^was a g^amblin' ^man
^Down in ^New Or^leans. ^
\end{verse}

\begin{instrumental}{Interludium}
\[Am] \[C] \[D] \[F]
\[Am] \[C] \[E7]
\end{instrumental}

\begin{verse}
Now the ^only ^thing a ^gambler ^needs
Is a ^suitcase ^and a ^trunk. ^
And the ^only ^time that ^he is satis^fied,
Is ^when he ^is all ^drunk. ^
\end{verse}


%% The first chorus is special in that it contains a "header" and sets up the memorization register
%% "chorus" for subsequent choruses. The header can be customized via the pgf key
%% "/gsb/chorusheading".
\begin{chorus}
There i\[X]s a h\[C]ouse in \[D]New Or\[F]leans
They \[Am]call the R\[Y]ising \[E7]Sun, \[E7]
And it's \[Am]been the ru\[C]in of m\[D]any a poor \[F]boy.
And \[Am]God, I \[E]know I'm \[Am]one. \[Z]
\end{chorus}

%% We can also give a verse a prefix:
\begin{verse}[Vierte Strophe:]
With ^one foot ^on the ^platform ^
And the ^other ^on the ^train, ^
I'm ^going ^back to ^New Or^leans
To ^wear that ^ball and ^chain. ^
\end{verse}
%% Also, replaying chords independently from verses is automatically possible

%% To just repeat the main chorus, we can use this:
\repeatchorus
%% Which actually just puts "(/gsb/chorusheading)" into a \musicnote.

%% The macros `\lrepeat` and `\rrepeat` produce a left and right repeat sign
%% (which looks a bit cleaner than the original \lrep and \rrep)
\begin{verse}
\lrepeat{} There i\[Am]s a h\[C]ouse in \[D]New Or\[F]leans
They \[Am]call the \[C]Rising \[E7]Sun, \[E7]
And it's \[Am]been the \[C]ruin of \[D]many a poor \[F]boy.
And \[Am]god, I \[E7]know I'm \[Am]one. \rrepeat{}
\end{verse}
 
%% We can add a number, if the chorus should be repeatedly repeated:
\repeatchorus[2]
\morespace
%% And this is one of the few cases where we need to _add_ space
%% (see below for \lessspace).

%% A chorus which is not the first one can also optionally be given a header:
\begin{chorus}[Auch ein Refrain:]
Oh ^mother ^tell your ^children ^
Not to ^do what ^I have ^done, ^
To ^spend your ^life in ^sin and mise^ry
In the ^House of the ^Rising ^Sun. ^
\end{chorus}

%% In case some text part contains no chords, the line spacing can be reduced so that 
%% the lines don't look so empty (and space is saved):
\begin{verse*}[Keine Akkorde:]
\singlespacing
Oh mother tell your children 
Not to do what I have done, 
To spend your life in sin and misery
In the House of the Rising Sun. 
\end{verse*}

\end{song}

%%%%%%%%%%%%%%%%%%%%%%%%%%%%%%%%%%%%%%%%%%%%%%%%%%%%%%%%%%%%%%%%%%%%%%%%%%%%%%%%%%%%%%%%%%%%%%%%%%%%
%% This is just another song, serving as a testcase for interaction between songs.

\begin{song}{In da Kinettn wo i schlof}[
  by={Wolfgang Ambros}]

\begin{instrumental}{Intro:}
\[D G D G E A D G E A]
\end{instrumental}

\begin{verse}
\[D] Wann in der Fruah die Nocht \brk gengan \[G]Tog den Kürzern ziagt.
\[D] Und waun da erste Sonnenstrahl \brk die letzte \[G]Dämmerung dawiagt.
\[E] Dann woch i \[A]auf \[D] \brk in der Ki\[G]nettn wo i \[E]schlof\ldots \[A]
\end{verse}

%% Here we see a bug: TeX says
%%   ERROR: Missing $ inserted.
%%   
%%   --- TeX said ---
%%   <inserted text> 
%%                   $
%%   l.136  ^
%%           Die Tschuschn kumman und \brk i muaß mi ^schleichn sonst zagns mi an.
%% without the {}; I probably messed up something with the active characters, and
%% the empty group fixes that.
\begin{verse}
{} ^ Die Tschuschn kumman und \brk i muaß mi ^schleichn sonst zagns mi an.
^ So kreul i hoit ausse und  \brk putz ma in ^Dreck ab, so guat i kann.
^ So steh i ^auf, ^ in der Ki^nettn wo i ^schlof,
so steh i ^auf\ldots
\end{verse}

\begin{verse}
{} ^ I hob mi scho seit zehn Tog \brk nimmer ra^siert und nimmer gwoschn,
^ Und i hob nix als wia a Flaschl \brk ^Rum in der Manteltaschen
^ Den gib i ma zum ^Frühstück \brk und dann ^schnorri an um ^a Zigaretten ^an,
Und um an ^Schilling.
\end{verse}

\begin{verse}
{} ^ Und de Leut kommen ma ent^geg'n \brk wia a Mauer kommens auf mi zua.
^ I bin der einzige der ihr ent^geg'n geht, \brk kummt ma vur
^ Oba i ^reiß mi zaum und ^moch \brk beim ersten ^Schritt die Augn ^zua ^
\end{verse}

\begin{instrumental}{Bridge:}
\[Asus4 A D G D G E A D G E A]
\end{instrumental}

\begin{verse}
{} ^ Es is do ganz egal \brk ob i was ^arbeit oder net, 
^ wei fia die dünne Klostersuppen \brk ge^nügts doch a waun i bet.
^ Laßts mi in ^Rua, weu \brk heit schüttns ^mei Kinettn ^zua.
Laßt's mi in ^Ruah!
\end{verse}

\end{song}

%%%%%%%%%%%%%%%%%%%%%%%%%%%%%%%%%%%%%%%%%%%%%%%%%%%%%%%%%%%%%%%%%%%%%%%%%%%%%%%%%%%%%%%%%%%%%%%%%%%%
% This example is just for comparison -- it shows how the above effects would be achieved with the
% original songs syntax, and that the new macros are backwards compatible:

\beginsong{Arschblender}[%
  by={Links\ Rechts\ Hilfe}]

\beginverse*
\musicnote{Verwendete Akkorde:}
\gtab{E7}{022130} \gtab{A}{X02220} \gtab{A7}{X02223}
\gtab{B7*}{7:131211} \gtab{A7*}{5:131211}
\musicnote{Intro:}
\[E E7 Asus4 A7] \chordcorrection
\endverse

\beginverse\memorize
\[E] Und jetzt \[E7]steh ich hier \brk am \[A]Häusl rum, \[A7]
\[E] um \[E7]mich sind lauter \brk \[A]Bräute rum, \[A7]
\[E] und ich \[E7]trage meine \brk \[A]Rockstar-Socken. \[A7]
\[E] Und \[E7]heute werd ich \brk \[A]richtig rocken \[A7]
\endverse

\beginchorus
\musicnote{Refrain:}
Denn die \[B7*]Sonne scheint mir \brk aus dem \[A7*]Arsch!
die \[B7*]Sonne scheint mir \brk aus dem \[A7*]Arsch,
und es geht mir so gut.
Ich bin ein \[E]Arschblender, \brk \[E7] ein Ver\[A]blendeter, \[A7]
ohh -- ein echter \[E]Arschblender. \[E7]
\[A]Fünftausend Megawatt \[A7]Sonnenlicht \brk aus dem {\[E E7]A}rsch; \[A A7]
aus dem \[E]Arsch. \[E7 A A7 G]
\endchorus

\beginverse
^ Und ^manche haben \brk ein ^Arschgeweih, ^
^ und ^manche haben ^zwei \brk \echo{oder ^drei!}
^ Und ^manche haben \brk einen ^Nasenring; ^
^ wer ^weiss, wo der \brk da^vor schon hing? ^
\endverse

\musicnote{(Refrain)}

\beginverse
^ Und ^manche machen \brk ^fette Beute; ^
^ das ^sind nicht immer \brk ^nette Leute. ^
^ Und ^manche haben so \brk ^ihre Macken, ^
^ und ^meine ist es, \brk ^Licht zu kacken! ^
\endverse

%% Often, two things are too far away; we can use \lessspace to fix that
%% (Which by default removes 0.5\baselineskip.)
\musicnote{(Refrain)} 
\lessspace
\musicnote{(Kazoo-Solo, mit \chord{E-E7-A-A7} begleitet)}

\beginverse*
Die \[E]Sonne scheint mir, \brk die \[E7]Sonne scheint mir,
die \[A]Sonne scheint mir \brk aus dem \[A7]A-arsch!
\[E]Na na na na, \[E7]da da da da, \brk \[A]di di di da \[A7]da da\ldots
Die \[E]Sonne scheint mir, \brk die \[E7]Sonne scheint mir,
die \[A]Sonne scheint mir \brk aus dem \[A7]A-arsch!
\[E]Na na na na, \[E7]da da da da, \brk \[A]di di di da \[A7]da da\ldots
\endverse

\musicnote{(Refrain)}

\endsong

\end{songs}

\end{document}
